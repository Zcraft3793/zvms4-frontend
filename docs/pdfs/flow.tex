\documentclass{article}
\usepackage[UTF8]{ctex}
\usepackage{amsmath}
\usepackage{amssymb}
\usepackage{listings}
\usepackage{geometry}
\usepackage{mathabx}
\newcommand{\varparallel}{\mathrel{/\mkern-5mu/}}
\geometry{a4paper,scale=0.8}
\usepackage[colorlinks,linkcolor=blue]{hyperref}
\usepackage{graphicx}
\usepackage{float}
\usepackage{subfigure}
\usepackage{multicol}
\usepackage{enumerate}
\usepackage{enumitem}
\usepackage{pifont}
\usepackage{ulem}
\usepackage{indentfirst}
\usepackage{tikz}
\usepackage{siunitx}
\usepackage[version=4]{mhchem}
\usepackage{chemmacros}
\usepackage{chemfig}
\usepackage{pifont}
\usepackage{url}
\usepackage[framemethod=tikz]{mdframed}
\chemsetup[phases]{pos=sub}
\usetikzlibrary{arrows,shapes,automata,petri,positioning,calc,shapes.geometric}

\tikzstyle{startstop} = [rectangle, rounded corners, minimum width=3cm, minimum height=1cm,text centered, draw=black]
\tikzstyle{io} = [trapezium, trapezium left angle=70, trapezium right angle=110, minimum width=3cm, minimum height=1cm, text centered, draw=black]
\tikzstyle{process} = [rectangle, minimum width=3cm, minimum height=1cm, text centered, draw=black]
\tikzstyle{decision} = [diamond, minimum width=3cm, minimum height=1cm, text centered, draw=black]
\tikzstyle{arrow} = [thick,->,>=stealth]

\title{ZVMS 4.0.0 操作流程}
\author{7086cmd}
\date{\today}

\begin{document}

\maketitle

对于 ZVMS 而言, 拥有一套完整的操作流程是非常重要的. 本文档将会介绍 ZVMS 的操作流程.

\tableofcontents

\newpage

\section{普通义工}

义工模块是 ZVMS 的核心模块. 该模块包含了所有的义工信息, 例如义工的时间, 地点, 时长等.

ZVMS 的义工分类请见用户文档. 本文档将会介绍义工的操作流程.

\subsection{指定义工}

\begin{mdframed}
  \fangsong
  指定义工是由管理员创建,发布为招募的义工,需要指定每个班级可以报名的人数上限。 \hfill 平台语料库 \\
  校内义工是学生参加学校组织的校园活动服务、劳动卫生、学生工作等活动。 \hfill 义工管理细则 \\
  校内义工由学生会实践部负责组织管理。在校园活动需要时,学生会各部门可以向全校征集义工,\textbf{具体内容应向学生会实践部提前报备},学生会成员未经报备,不得向全校征集义工。\textbf{义工时间发放时长应由实践部视具体情况核准,否则一律按最低标准发放。} \hfill 义工管理细则
\end{mdframed}

指定义工分为以下三种 (不包括补录):

\begin{enumerate}
  \item \textbf{正常计划招募义工}: 该义工为实践部组织, 团支书需要提前安排好相关的人员, 并且按照平台中的指定时间地点进行义工服务.
  \item \textbf{教师安排临时义工}: 该义工为教师临时安排, 需要提供专门的证明材料, 由负责教师和组长签名后方可有效.
  \item \textbf{大型校内活动义工}: 该义工为实践部要求班级策划, 统一发放时间.
\end{enumerate}

\subsubsection{正常计划招募义工}

正常计划招募义工由实践部发布. 若情况紧急可由团支书创建, 但需要实践部审核通过后方可生效.

\begin{figure}[H]
  \centering
  \begin{tikzpicture}[node distance=10pt]
    \node (start) [startstop] {开始};
    \node (input) [io, below=of start] {创建义工};
    \node (check) [decision, below=of input] {是否为团支书创建};
    \node (audit) [process, right=of check] {实践部审核};
    \node (activity) [process, below=of check] {义工活动};
    \node (reflect) [process, below=of activity] {感想填写和感想反馈};
    \node (end) [startstop, below=of reflect] {结束};

    \draw [arrow] (start) -- (input);
    \draw [arrow] (input) -- (check);
    \draw [arrow] (check) -- node[anchor=east] {是} (audit);
    \draw [arrow] (check) -- node[anchor=south] {否} (activity);
    \draw [arrow] (audit) -- (activity);
    \draw [arrow] (activity) -- (reflect);
    \draw [arrow] (reflect) -- (end);
  \end{tikzpicture}
  \caption{正常计划招募义工流程}
  \label{fig:normal}
\end{figure}

\subsubsection{教师安排临时义工}

教师安排临时义工由教师发布. 该义工需要提供专门的证明材料, 由负责教师和组长签名后方可有效. (证明文件为 \textit{ZVMS 4 临时义工证明}.pdf).

\begin{figure}[H]
  \centering
  \begin{tikzpicture}[node distance=10pt]
    \node (start) [startstop] {开始};
    \node (do) [process, below=of start] {教师安排义工};
    \node (proof) [process, below=of do] {创建证明文件};
    \node (secretary) [io, right=of proof] {团支书创建};
    \node (audit) [process, below=of secretary] {实践部审核};
    \node (practice) [io, left=of proof] {实践部创建};
    \node (reflect) [process, below=of practice] {感想填写和感想反馈};
    \node (end) [startstop, below=of reflect] {结束};

    \draw [arrow] (start) -- (do);
    \draw [arrow] (do) -- (proof);
    \draw [arrow] (proof) -- node[anchor=south] {\ding{172}} (secretary);
    \draw [arrow] (secretary) -- (audit);
    \draw [arrow] (audit) -- (reflect);
    \draw [arrow] (proof) -- node[anchor=south] {\ding{173}} (practice);
    \draw [arrow] (practice) -- (reflect);
    \draw [arrow] (reflect) -- (end);
  \end{tikzpicture}
\end{figure}

其中, \ding{172}, \ding{173} 为任意选择其中一者操作. 若寻找实践部成员创建, 应寻找负责本班或本办公室的实践部成员进行创建. 具体视实践部情况而定.

\subsubsection{大型校内活动义工}

大型校内活动由学校发起, 各班提供策划案并统一发放义工. 寒暑假期间的护校义工等也属于此类.

原则上该义工活动单个记入单条义工纪录. 可以在创建时提供相关附件 (例如策划案, 证明文件等, \textit{正在开发中}).

因此, 大型校内活动义工的流程与正常计划招募义工类似, 不再赘述.

\subsubsection{感想填写和审核流程}

所有的校内指定义工都需要填写感想, 由学生义工自主管理委员会的审计部负责审核.

在 \texttt{ZVMS 4} 中, 团支书的初审对于平时的校内校外活动已经取消. 但是对于寒暑假大型社会实践的优秀实践选择, 仍然需要团支书初审.

\begin{figure}[H]
  \centering
  \begin{tikzpicture}[node distance=10pt]
    \node (start) [startstop] {开始};
    \node (activity) [process, below=of start] {完成义工活动};
    \node (fill) [io, below=of activity] {填写感想};
    \node (save) [process, right=of fill] {保存感想};
    \node (submit) [process, below=of fill] {提交感想};
    \node (audit) [io, below=of check] {审计部审核};
    \node (effective) [process, below=of audit] {审核通过};
    \node (rejected) [process, left=of audit] {感想被驳回};
    \node (refused) [process, right=of audit] {感想被拒绝};
    \node (refuse) [startstop, below=of refused] {不得发放义工时间};
    \node (end) [startstop, below=of effective] {发放义工时间};

    \draw [arrow] (start) -- (activity);
    \draw [arrow] (activity) -- (fill);
    \draw [arrow] (fill) -- (save);
    \draw [arrow] (save) -- (fill);
    \draw [arrow] (fill) -- (submit);
    \draw [arrow] (submit) -- (audit);
    \draw [arrow] (audit) -- node[anchor=east] {通过} (effective);
    \draw [arrow] (audit) -- node[anchor=north] {驳回} (rejected);
    \draw [arrow] (audit) -- node[anchor=north] {拒绝} (refused);
    \draw [arrow] (refused) -- (refuse);
    \draw [arrow] (effective) -- (end);
    \draw [arrow] (rejected) -- (fill);
  \end{tikzpicture}
  \caption{感想填写和审核流程}
  \label{fig:reflect}
\end{figure}

\begin{mdframed}
  \fangsong
  任何人不得通过造假、冒名顶替等手段获取义工时间,各班团支书应做好监督检查工作,认真审核。
  违规行为按情节严重程度,经团委核准,处以取消当次义工时间、倒扣义工时间等处罚。违反校纪校规的,按照相关规定给予相应处分。
  \hfill \textit{义工管理细则}
\end{mdframed}

\subsection{社会义工}

\begin{mdframed}
  \fangsong
  社会义工是由团支书以上的人员可以创建,创建时需要指定所有参与人员,仅限于在学期内周末节假日校外的义工。 \hfill 平台语料库 \\
  校外义工是学生利用周末、小长假等课余时间在博物馆、体育馆、展览馆、医院、养老院、福利院、社区、农村、工厂等社会服务场所进行的志愿服务活动。 \hfill 义工管理细则 \\
  校外义工由各班团支书组织管理,原则上要求4至8人为单位组队完成。活动中各组要保质保量完成工作并进行整理报告。团支书应在每次活动前做好动员组织、分配队伍、数据统计等工作,在活动后收集、汇总各组报告,确认材料属实,并统一收集好以照片形式上传至义工管理平台。 \hfill 义工管理细则
\end{mdframed}

社会义工的填报流程与校内义工类似, 但是有以下几点不同:

\begin{itemize}
  \item \textbf{需要上传图片}: 上传图片是为了证明义工的真实性. 上传图片的格式为 \texttt{.jpg}, \texttt{.jpeg}, \texttt{.png}, \texttt{.bmp} 等格式. 文件将会上传至 \texttt{backblaze} 服务器, 在上传前会提请自动图片审核, 并压缩至 \texttt{3 MiB} 以内.
  \item \textbf{可以由学生自提交}: 社会义工可以由学生自行提交, 但是需要团支书审核通过后方可生效.
  \item \textbf{团支书创建无需审核}: 团支书创建的社会义工无需经过实践部审核.
  \item \textbf{有最大时间限制}: \textit{校外义工时间原则上一次不得超过5小时。} \hfill 义工管理细则
\end{itemize}

因此, 社会义工的流程与校内义工类似.

\begin{figure}[H]
  \centering
  \begin{tikzpicture}[node distance=10pt]
    \node (start) [startstop] {开始};
    \node (create) [io, below=of start] {学生创建义工};
    \node (audit) [process, below=of create] {团支书审核};
    \node (secretary) [io, right=of create] {团支书创建};
    \node (activity) [process, below=of audit] {义工活动};
    \node (image) [io, below=of activity] {上传图片};
    \node (reflect) [process, below=of image] {感想填写和感想反馈};
    \node (end) [startstop, below=of reflect] {结束};

    \draw [arrow] (start) -- (create);
    \draw [arrow] (start) -- (secretary);
    \draw [arrow] (create) -- (audit);
    \draw [arrow] (secretary) -- (activity);
    \draw [arrow] (audit) -- (activity);
    \draw [arrow] (activity) -- (image);
    \draw [arrow] (image) -- (reflect);
    \draw [arrow] (reflect) -- (end);
  \end{tikzpicture}
  \caption{社会义工感想填写和审核流程}
\end{figure}

在审核时, 审计部成员将会对图片进行审核, 并且对感想进行审核. 若审核通过, 则发放义工时间.

\subsection{实践义工}

\begin{mdframed}
  \fangsong
  实践义工是由任何人都可以创建,创建时需要指定所有参与人员,仅限于在寒暑假期间根据学校文件的社会实践。 \hfill 平台语料库 \\
  大型社会实践义工是学生在寒暑假期参加学校组织校外社会实践活动。 \hfill 义工管理细则 \\
  大型社会实践义工由学生会实践部组织,各班同学参与,原则上每次活动每位同学应至少参加一项。 \hfill 义工管理细则
\end{mdframed}

社会实践义工的提交目前仍在商讨阶段. 目前共有如下几个阶段 (\texttt{Stage}):

\begin{enumerate}
  \item 实践部成员酝酿社会实践主题, 构想活动形式并攥写相关策划案, 并录入平台. \hfill Stage 1
  \item 实践部成员向学生会和团委提交策划案, 学生会和团委审核通过后, 向全校发布, 并在平台上开启填报入口. \hfill Stage 2
  \item 学生组织进行实践活动, 选择其中的至少一项 (单条记录对应一项), 可重复创建. \hfill Stage 3
  \item 学生在寒暑假完成相关活动, 在平台或平台的钉钉小程序插件中提交相关材料. \hfill Stage 4
  \item 团支书初步审核, 选择优秀社会实践, 全部提请审核. \hfill Stage 5
  \item 审计部审核, 发放义工时间. \hfill Stage 6
\end{enumerate}

在审核时, 分为 \textit{优秀}, \textit{合格}, \textit{不合格} 三种状态. 三者皆会发放义工时间, 但是相关时间见下表:

\begin{table}[H]
  \centering
  \begin{tabular}{cc}
    \hline
    \textbf{状态} & \textbf{时间} \\
    \hline
    优秀 & $9.0$ 小时 \\
    合格 & $6.0$ 小时 \\
    不合格 & $3.0$ 小时 \\
    \hline
  \end{tabular}
  \caption{社会实践义工时间}
  \label{tab:practice}
\end{table}

\section{获奖页面}

\begin{mdframed}
  \fangsong
  个人获奖类义工总时长不超过10小时,可由个人选择计入校内义工时长或者校外义工时长。若以集体为单位参赛的,视具体情况分层发放。 \hfill 义工管理细则 \\
  获奖者可在每次获奖时选择记入校内或校外义工时间,记入后不可更改获奖信息。 \hfill 平台语料库
\end{mdframed}

从 \texttt{ZVMS 4} 开始, 获奖的义工时间分发将会启动全新的页面. 该页面将会展示所有的获奖信息, 并且可以进行相关的操作.

在 \texttt{ZVMS 4} 中, 拥有 \textit{学术类}, \textit{体育类}, \textit{艺术类}, 和 \textit{其他} 四种获奖类型. 每种获奖类型都有不同的时间分发规则. 同时, 根据规模分为 \textit{区级}, \textit{市级}, \textit{省级}, \textit{国家级}, \textit{国际级} 五种规模.

义工时间的分发原则上根据 \textit{义工管理细则} 进行分发, 具体视实践部讨论而定.

获奖的创建原则上必须由实践部成员进行. 若团支书创建, 则需要实践部审核通过后方可生效.

\begin{figure}[H]
  \centering
  \begin{tikzpicture}[node distance=10pt]
    \node (start) [startstop] {开始};
    \node (competition) [process, below=of start] {举办某项竞赛};
    \node (secretary) [io, left=of competition] {团支书创建};
    \node (practice) [io, right=of competition] {实践部创建};
    \node (audit) [process, below=of secretary] {实践部审核};
    \node (collect) [process, below=of competition] {学生填报获奖信息};
    \node (if) [decision, below=of collect] {是否有指导老师};
    \node (proof) [process, below=of if] {学生提供书面证明材料};
    \node (reflect) [process, right=of if] {实践部成员寻找相关指导老师进行审核};
    \node (end) [startstop, right=of proof] {发放义工时间};

    \draw [arrow] (start) -- (competition);
    \draw [arrow] (competition) -- (secretary);
    \draw [arrow] (competition) -- (practice);
    \draw [arrow] (secretary) -- (audit);
    \draw [arrow] (practice) -- (collect);
    \draw [arrow] (audit) -- (collect);
    \draw [arrow] (collect) -- (if);
    \draw [arrow] (if) -- node[anchor=south] {否} (proof);
    \draw [arrow] (if) -- node[anchor=south] {是} (reflect);
    \draw [arrow] (proof) -- (end);
    \draw [arrow] (reflect) -- (end);
  \end{tikzpicture}
\end{figure}

教师审核文件的格式为 \textit{ZVMS 4 获奖证明}.pdf. 证明材料应提供书面证明, 且应包含相关的获奖信息, 并附上 \textit{ZVMS 4 个人获奖证明}.pdf 文件.

获奖义工时间分发无需填写感想, 审核有效后即可发放义工时间.

\section{特殊义工}

除去获奖义工, 特殊义工分为以下几种:

\begin{itemize}
  \item 社团类特殊义工
  \item 导入类获奖义工
  \item 扣除义工时间
  \item 其他特殊义工
\end{itemize}

\subsection{社团类获奖义工}

社团类获奖义工需要指导老师和社团负责人 (团长) 签名的 \textit{ZVMS 4 社团成员义工时间分级发放证明}.pdf 文件.

\begin{figure}[H]
  \centering
  \begin{tikzpicture}[node distance=10pt]
    \node (start) [startstop] {开始};
    \node (September) [process, below=of start] {九月份社团成员义工时间分级发放证明};
    \node (sign) [process, below=of September] {指导老师和社团负责人签名};
    \node (submit) [process, below=of sign] {提交至实践部};
    \node (audit) [process, below=of submit] {实践部审核};
    \node (end) [startstop, below=of audit] {发放义工时间};

    \draw [arrow] (start) -- (September);
    \draw [arrow] (September) -- (sign);
    \draw [arrow] (sign) -- (submit);
    \draw [arrow] (submit) -- (audit);
    \draw [arrow] (audit) -- (end);
  \end{tikzpicture}
\end{figure}

\subsection{导入数据}

导入数据需要提供相关的证明材料: \textit{ZVMS 4 数据导入}.xlsx 和 \textit{ZVMS 4 数据导入申请}.pdf, 并且由 \textit{ZVMS 4 数据导入模板}.xlsx 为模板进行填写.

\begin{figure}[H]
  \centering
  \begin{tikzpicture}[node distance=10pt]
    \node (start) [startstop] {开始};
    \node (import) [process, below=of start] {实践部决定导入数据};
    \node (collect) [process, below=of import] {审计部统计需要导入的数据};
    \node (submit) [process, below=of collect] {将数据交由平台管理员};
    \node (end) [startstop, below=of submit] {平台管理员导入并发放义工时间};

    \draw [arrow] (start) -- (import);
    \draw [arrow] (import) -- (collect);
    \draw [arrow] (collect) -- (submit);
    \draw [arrow] (submit) -- (end);
  \end{tikzpicture}
\end{figure}

\subsection{扣除义工时间}

过程正在商讨.

\subsection{其他特殊义工}

原则上不得使用其他特殊义工. 若有特殊情况, 请联系实践部成员.

\end{document}
